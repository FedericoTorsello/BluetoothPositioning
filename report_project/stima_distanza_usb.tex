\chapter{Stima della distanza con Arduino}
\section{Progetto per stimare la distanza con un sensore ultrasonico ed un Arduino}
Per poter stimare la distanza con Arduino è stato utilizzato un sensore ultrasonico. Questo sensore è stato connesso direttamente alla board attraverso cavi maschio-femmina.

\begin{lstlisting}[language=XML]
<intent-filter>
	<action android:name="android.intent.action.MAIN" />
	<action android:name="android.hardware.usb.action.USB_DEVICE_ATTACHED"/>

	<category android:name="android.intent.category.DEFAULT" />
	<category android:name="android.intent.category.LAUNCHER" />

</intent-filter>

<meta-data android:name="android.hardware.usb.action.USB_DEVICE_ATTACHED" android:resource="@xml/device_filter" />
\end{lstlisting}



\begin{itemize}
	 \item Implementare un progetto Arduino che faccia uso della libreria \href{http://playground.arduino.cc/Code/NewPing}{\textbf{NewPing}}\footnote{\href{http://playground.arduino.cc/Code/NewPing}{\textbf{NewPing}} - \url{http://playground.arduino.cc/Code/NewPing}};
	 
	 \item Sfruttare la libreria \href{https://github.com/mik3y/usb-serial-for-android}{\textbf{usb-serial-for-android}}\footnote{\href{https://github.com/mik3y/usb-serial-for-android}{\textbf{usb-serial-for-android}} - \url{https://github.com/mik3y/usb-serial-for-android}} per connettere l'Arduino ad Android e ricevere i dati seriali in tempo reale.
	 
	 \item Sviluppare un mini progetto su \href{https://netbeans.org/}{\textbf{NetBeans}}\footnote{\href{https://netbeans.org/}{\textbf{NetBeans}} - \url{https://netbeans.org/}} ed impiegare la libreria \href{https://github.com/scream3r/java-simple-serial-connector}{\textbf{jSSC}}\footnote{\href{https://github.com/scream3r/java-simple-serial-connector}{\textbf{jSSC}} - \url{https://github.com/scream3r/java-simple-serial-connector}} (\textit{java-simple-serial-connector}) per testare il sensore di prossimità collegando l'Arduino al PC e visualizzando la distanza a schermo.
\end{itemize}

\subsubsection{\underline{\href{http://playground.arduino.cc/Code/NewPing}{NewPing}}}\label{sec:newping}
\textbf{Caratteristiche}
\begin{itemize}
	\item Compatibile con diversi modelli di sensori ad ultrasuoni: SR04, SRF05, SRF06, DYP-ME007 e Parallax Ping™.
	
	\item Non soffre di \textbf{lag} di un secondo se non si riceve un ping di eco.
	
	\item Produce un ping coerente e affidabile fino a 30 volte al secondo.
	
	\item Timer interrupt method per sketch event-driven.
	
	\item Metodo di filtro digitale built-in \texttt{ping\_median()} per facilitare la correzione degli errori.
	
	\item Utilizzo dei registri delle porte durante l'accesso ai pin per avere un'esecuzione più veloce e dimensioni del codice ridotte.
	
	\item Consente l'impostazione di una massima distanza di lettura del ping "in chiaro".
	
	\item Facilita l'utilizzo di più sensori.
	
	\item Calcolo distanza preciso, in centimetri, pollici e uS.
	
	\item Non fa uso di \texttt{pulseIn}, metodo che risulterebbe lento e che con alcuni modelli di sensore a ultrasuoni restituisce risultati errati.
	
	\item Attualmente in sviluppo, con caratteristiche che vengono aggiunte e bug/issues affrontati.
\end{itemize}

\section{Progetto di test della comunicazione seriale PC - Arduino}
Il metodo più importante di questo mini progetto è \texttt{updateDistance()}. Questo metodo è di tipo \textbf{void}. Il suo scopo è quello di riceve in input i \textit{byte} dalla porta seriale e convertirli in stringhe da visualizzare a schermo su una \textit{jLabel}.

Per poter fare I/O sulla porta seriale si deve istanziare e configurare l'oggetto SerialPort, abilitando la comunicazione via USB con una board Arduino.

Per avviare la comunicazione:
\begin{itemize}
	\item si settano i parametri di ingaggio (baund rate, numero di bit dei pacchetti, numero dei bit di stop e se è presente un controllo di parità)
	
	\item si imposta l'event mask in modo da controllare se sul canale sono prenti \textit{char}
	
	\item si registra l'istanza di SerialPort in un listener di eventi di I/O della seriale per poi considerare solo i dati di tipo \textit{char} e scartare tutti gli altri. 
\end{itemize}


Metodo updateDistance()
\begin{lstlisting}[language=Java]
private void updateDistance() {    
	SerialPort serialPort = new SerialPort("/dev/ttyACM0");
	try {
		serialPort.openPort();
		
		serialPort.setParams( 
			SerialPort.BAUDRATE_115200, 
			SerialPort.DATABITS_8, 
			SerialPort.STOPBITS_1,
			SerialPort.PARITY_NONE);
			
		serialPort.setEventsMask(SerialPort.MASK_RXCHAR);
		
		serialPort.addEventListener((SerialPortEvent serialPortEvent) -> {
			if (serialPortEvent.isRXCHAR()) {
				try {
					Thread.sleep(20);
					String distance = serialPort.readString();
					jLabel1.setText(distance);
				} catch (SerialPortException | InterruptedException ex) {
				}
			}
		});
	} catch (SerialPortException ex) {
		System.out.println("SerialPortException: " + ex.toString());
	}
}
\end{lstlisting}

