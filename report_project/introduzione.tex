\chapter{Introduzione}

Stimare con precisione la distanza che intercorre tra un utente ed un punto target è, secondo molti, una sfida tecnologica. Sempre più sistemi	 con applicazioni mobile e non necessitano di questa informazione per poter funzionare in modo corretto. 
	
Per stimare la distanza esistono diverse tecnologie più o meno precise e costose. In questo senso l'avvento delle tecnologie IoT (\textit{Internet of things}) ha influito positivamente abbassando il costo dell'hardware necessario, creando community di hobbisti e professionisti, quindi ampliando il numero di librerie software (spesso open source) disponibili e progetti da cui prendere spunto.
	
\section{Vision}

\begin{itemize}
	\item Realizzare un sistema software mobile per stimare al meglio la distanza che intercorre tra l'utente e gli iBeacon disposti in un ambiante indoor.
	
	\item Utilizzare solo tecnologie open source per realizzare il tutto, ribadendone l'utilità e l'efficienza.
\end{itemize}

\section{Goals}
\subsection{\underline{Goals principali}}
\begin{enumerate}
	\item Sviluppare un'app Android in grado di interagire con degli iBeacon disposti in una stanza.
	
	\item Realizzare un'app compatibile con tutte le API Android 18 e superiori.
	
	\item L'app deve utilizzare i valori RSSI dei iBeacon per determinare la distanza trasmettitore-ricevitore.
	
	\item Implementare diversi filtri per ridurre gli effetti indesiderati del \textit{multipath fading} sulla stima della distanza.
	
	\item Realizzare e testare un \textbf{filtro di Kalman}, un \textbf{filtro ARMA} (\textit{Auto Regressive Moving Average}) e un \textbf{filtro RunningAverageRssi} per limitare gli effetti indesiderati sopracitati.
	
	\item Visualizzare dei grafici sull'app che in tempo reale descrivano l'andamento della distanza stimata.
	
	\item Curare la \textit{User Experience} realizzando una GUI chiara e responsive utilizzando librerie \textit{com.android.support}
	
	\item Sviluppare l'intero sistema utilizzando GNU/Linux e FOSS (Free and open-source software).
\end{enumerate}

\subsection{\underline{Goals secondari}}
\begin{enumerate}	
	\item Realizzare un programma C++/Wiring in grado di determinare la distanza percepita da un sensore ultrasonico collegato ad una board Arduino.
	
	\item Realizzare un mini progetto per testare il sensore ultrasonico. Nello specifico si considera un Arduino connesso al PC attraverso la porta USB e una view di feedback per visualizzare a schermo la distanza "reale".
	
	\item Abilitare la comunicazione seriale mediante tecnologia \textbf{USB OTG}, facendo diventare lo smartphone un host USB.
	
	\item Implementare la visualizzazione della distanza percepita dall'Arduino direttamente nell'app. L'obiettivo è dare un feedback della reale distanza iBeacon-utente e poterla mettere a confronto con quella stimata utilizzando gli RSSI. 
	
	In questo caso si considera un \textbf{Arduino connesso allo smartphone} attraverso la porta USB.
\end{enumerate}



