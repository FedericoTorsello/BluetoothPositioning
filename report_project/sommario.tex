\begin{abstract}

\begin{description}
	\item [Capitolo 1] In questo capitolo si viene introdotti al progetto, quindi vengono descritti la vision e i goals.
	
	\item [Capitolo 2] In questo capitolo si definiscono ed analizzano i requisiti funzionali e non funzionali, i casi d'uso e gli scenari relativi al progetto.
	
	\item [Capitolo 3] In questo capitolo si analizza brevemente la tecnologia iBeacon ed in generale il BLE (Bluetooth Low Energy). Sono inoltre forniti alcuni esempi di utilizzo reali e caratteristiche del protocollo di comunicazione.
	
	\item [Capitolo 4] In questo capitolo si definisce cos'è RSSI, come calcolarlo e come sfruttarlo per stimare la distanza utente-iBeacon target.
	
	\item [Capitolo 5] In questo capitolo si affronta la stima della distanza con RSSI, Android e tecnologie Bluetooth Low Energy. In particolare si introduce all'utilizzo della libreria AltBeacon e di filtri su RSSI e sulle distanze stimate.
	
	\item [Capitolo 6] In questo capitolo si parla della stima della distanza utilizzando Arduino, prima con un progetto di test con questo connesso al PC e poi con l'implementazione legata all'app. Nel dettaglio si parlerà di come è possibile connettere direttamente l'Arduino allo smartphone per ricevere dati sensoristici.
	
	\item [Capitolo 7] In questo capitolo si struttura il progetto facendo l'analisi delle classi, elencando alcuni codici sorgente di esempio.
	
	\item [Capitolo 8] In questo capitolo si passa all'implementazione vera e propria dell'app, riportando immagini e descrizioni del suo utilizzo.
	
	\item [Capitolo 9] In questo capitolo si definiscono le condizioni e gli strumenti per realizzare il testing reale della stima della distanza.
	
	\item [Capitolo 10] In questo capitolo si analizzano dei casi di testing reali e si traggono delle conclusioni sulla valenza delle tecniche di stima utilizzate.
\end{description}

\end{abstract}